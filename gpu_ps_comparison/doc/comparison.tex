\documentclass{article}
\usepackage{graphicx}
\usepackage{epsfig}
\usepackage{tikz}
\usetikzlibrary{arrows,decorations}
\usetikzlibrary{decorations.markings}
\usepackage{listings}
\usepackage{minted}
\usepackage{array} % for better arrays (eg matrices) in maths

\begin{document}

\title{Speed Comparison of Three Simulation Methods using GLSL}
\author{Jeffrey Bowles\\
  Department of Computer Science \\
  University of New Mexico\\
  \texttt{jbowles@riskybacon.com}}
\maketitle

\section{Introduction}
It is currently very popular to perform simulations on a Graphics Processing Unit (GPU) using CUDA or OpenCL. These tools are very nice to use as they do not require a knowledge of OpenGL, nor do they require that a GL context be created to use the GPU. However, application developers that have a working knowledge of OpenGL may want to perform a simulation and use the results in an OpenGL program. These developers are already familiar with using OpenGL function calls to move data on and off the GPU and are already familiar with writing shader programs on the GPU. These application developers do not need CUDA or OpenCL to perform these simulations, and they may not want to learn a new set of terminology, syntax and idioms to get the job done.

A quick search of the literature shows that there is more than one way to perform simulations using OpenGL and GLSL, but it isn't clear which method is the fastest. This paper discusses three different methods and their benchmarks. The methods used are Transform Feedback, Vertex Texture Fetch, and Pixel Buffer Object. 


\section{Methodology}
All methods use two passes. The first pass computes the new
positions of the particles and the second pass renders the
particles. Each method was written to be as similar as possible to the other methods to ensure a fair comparison
The tests were run on a 17" MacBook Pro,
2.3GH Intel Core i7, 8GB memory, AMD Radeon HD 6750M with 1G memory.

An OpenGL 3.2 core profile was used, with GLSL 1.5.

In all three cases, RK4 is used to determine new particle positions
and velocities. The simulation is that of a gravity well. Half of the
particles are started at (0, 0.1, 0) and the other half at (0, -0.1, 0).
Initial velocities all have the same magnitude, all particles have the
same mass, and a gravity well exists at (0,0,0). While all particles
have the same velocity magnitude, their directions are random.

\section{Transform Feedback}

This method was the slowest, but also the easiest to understand. The Transform feedback
buffer is used in the first pass to update particle positions. This is
the easiest method to reason about because the positions are 
always stored as a list of vertices. However, it was the hardest
to develop because there aren't any good examples of how to use
the transform feedback buffer in OpenGL 3.2. 

\section{Vertex Texture Fetch}

This method was the fastest. Particle positions and velocities
are stored in two RGBA floating point texture maps and are
attached to an FBO. Two FBOs are created and "ping-ponging" between
the two for each update and render pass is used to avoid writing
to the same texture map that is being read.

Particle system updates are performed in the fragment shader and 
Multiple Render Targets (MRT) are used to write both the new position
and the new velocity.

A Vertex Array Object (VAO) is that has a buffer object attached
that contains a vertex for each particle that is represented in the
texture map. Each vertex has a single, static attribute. This attribute
is the texture coordinates used to index into the particle position
texture map. It is important to generate texture coordinates that
point to the center of the texel.

In the vertex shader, the texture coordinate attribute is used
to read the particle's position out of the position texture map.

\section{Pixel Buffer Object (PBO)}

This was the middle-of-the road solution. As with the vertex
texture fetch solution, two FBOs are created with position
and velocity texture attached to the outputs. 


Particle system update are performed in the fragment shader and
MRTs are used to write both the new position and the new velocity.

As with the VTF method, a VAO is created with a buffer object
that contains a vertex for each particle. The difference is that
instead of texture coordinates as the attribute for each particle,
the attribute will be the position of the particle. This eliminates
the need to perform a texture lookup in the vertex shader. However,
it does require a copy operation by the GL.

It is hoped that this copy operation will be a pointer change, and
not an actual copy, but the results indicate that a copy occurs.

Using the pixel pack buffer, the positions in the texture map
are copied into the buffer object. One might assume that the
best performance would be acheived by marking the buffer object
as GL\_STREAM\_COPY during the glBufferData() call, but through trial
and error it was discovered that GL\_STATIC\_DRAW was by far the
best method.


\section{Results}

Each version of the simulation was run for 1000 steps. Each particle was rendered as 
a point sprite with the size specified as glPointSize(1.0f). The number of particles is represented in millions and the frames per second were measured.

\begin{tabular}{llll}
Particles	&	Transform Feedback		& Vertex Texture Fetch &	Pixel Buffer Object \\
\hline
0.25		&	200	&	500	&	333 \\
0.50		&	100	&	333	& 	250 \\
0.75		&	50		&  250	&	167	\\
1			&  40		&  200	&	125	\\
2			&  22		&  100	&	71 	\\
3			& 13		&   67    &	50		\\
4            & 10		&   52	&	41.7	\\
5			&  8		&	41.7	&	29.4	\\
6			&  6.8	&	33.3	&	24.4	\\
7			&	5.7	&	29.4	&	20.8	\\
8			&	5.1	&	26.3	&	18.5	\\
9			&	4.5	&	22.7	&	16.4	\\
10			&	4		&	20.8	&	14.7	\\
11			&	3.7	&	19.23	&	7.1	\\
12			&	3.4	&	16.7		&	6.25	\\
13			&	3.2	&	16.1		&  failed	\\
14			&	2.9	&	failed	&	failed	\\
15			&	2.7	&	failed	&	failed	\\
\end{tabular}

Failure was defined by runs that failed to produce any meaningful output, such as a black screen or garbled output. I believe that the issue is round-off error which caused a failure to fetch and update the proper texels.

\section{Future Work}
The performance tests should be ported to other systems. This should not be too difficult as the dependencies are all cross platform.

\end{document}